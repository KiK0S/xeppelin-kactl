% File: content/misc/BitOps.tex
% Short cheatsheet for GCC/Clang bit builtins & tricks
% Fits KACTL style: very compact, single small box.
\begin{center}
\small
\textbf{Bit builtins \& tiny hacks} \\
\texttt{// count set bits} \\
\texttt{int pop = \_\_builtin\_popcount(x);} \\
\texttt{// count leading zeros} \\
\texttt{int lz = \_\_builtin\_clz(x);} \\
\texttt{// count trailing zeros} \\
\texttt{int tz = \_\_builtin\_ctz(x);} \\
\texttt{// find first set (1-based)} \\
\texttt{int i = \_\_builtin\_ffs(x);} \\
\texttt{// parity (1 if odd \# of 1-bits)}  \\
\texttt{int p = \_\_builtin\_parity(x);}

\textbf{Common idioms (assume unsigned integer types):} \\[3pt]
\texttt{// isolate lowest set bit} \\ \texttt{unsigned lowbit = x \& -x;} \\[3pt]
\texttt{// turn off lowest set bit} \\ \texttt{x \&= x - 1;} \\[3pt]
\texttt{// index (0-based) of lowest set bit} \\ \texttt{int idx = \_\_builtin\_ctz(x);} \\[3pt]
\texttt{// index (0-based) of highest set bit (32-bit)} \\
\texttt{int hi = 31 - \_\_builtin\_clz(x);} \\[3pt]
\texttt{// highest power of two $\le x$ (32-bit)} \\
\texttt{unsigned hp = 1u << (31 - \_\_builtin\_clz(x));} \\[6pt]

Use the \texttt{ll} suffix for 64-bit integers (e.g. \texttt{\_\_builtin\_popcountll}).\\
\end{center}
